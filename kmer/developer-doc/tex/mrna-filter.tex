\documentclass[twoside]{article}
\usepackage{amsmath,amssymb}
\usepackage{moreverb}
\usepackage{fancyheadings}
\usepackage{ulem}
\usepackage{parskip}
\usepackage{calc,ifthen,epsfig}
\sloppy

\begin{document}

%  See page 63-64, LaTeX Companion
%
% leftmargin controls the left margin for EVERYTHING in the list!
%
\newcommand{\entrylabel}[1]{\mbox{\texttt{#1:}}\hfil}
\newenvironment{entry}
  {\begin{list}{}%
   {\renewcommand{\makelabel}{\entrylabel}%
    %\setlength{\leftmargin}{1.5in}%
   }}
{\end{list}}

% The first parbox width controls the indent on the first text line
% The makebox width seems to do nothing.
\newcommand{\Lentrylabel}[1]{%
  {\parbox[b]{0pt}{\makebox[0pt][l]{\texttt{#1:}}\\}}\hfil\relax}
\newenvironment{Lentry}
  {\renewcommand{\entrylabel}{\Lentrylabel}\begin{entry}}
  {\end{entry}}

\title{Filtering mRNA signal}
\author{Brian P. Walenz}
\maketitle

\pagestyle{fancy}

\rhead[Brian Walenz]{Brian Walenz}
\chead[Filtering mRNA signal]{Filtering mRNA signal}
\lhead[\today]{\today}

\normalem

\begin{abstract}
Given signals detected by chaining 20-mers, a method is presented for
deciding which signals potentially contain real matches.
\end{abstract}

\section{What is a signal}

Signal has three values associated with it.  The amount 'covered', the
amount 'matched' and the total 'length'.
%
The amount covered is the number of
bases in the mRNA that are contained in least one mer.
%
The amount matched is the number of paired bases (for example, position
$i$ in the cDNA paired with position $j$ in the genomic) covered by a mer.
%
The length is the number of mers in the mRNA (roughly equivalent to the
number of bases in the mRNA that could be covered by a mer, but easier
to compute).

From these, we can derive two scores, the coverage and the multiplicity.
The coverage, $\frac{covered}{length}$, represents the fraction of the mRNA
that we found, while the multiplicity, $\frac{matched}{covered}$, represents
the amount of the mRNA that we found too many times.

A high multiplicity usually indicates a repeat-containing mRNA.  High
multiplicity and high coverage can indicate that the mRNA is not cDNA.

\section{Filter}

In order to filter signals, we need to decide, for each mRNA, which
signals are bad, and which are good (duh!), which means that we'll
need to look at {\em all} signals for a single mRNA.

For the filter presented below, we need to know the best and worst
coverage values that occur for any signal associated with a specific
mRNA.  Once those are known, the signals can be filtered in any order.
This is important in the case where the signals are detected
chromosome by chromosome.  Instead of sorting all signals, we can save
the best and worst coverage for each mRNA.

\begin{figure}
\begin{center}
\begin{tabular}{|c|c|c|l|}
\hline
Switch & Variable & Default Value & Description \\
\hline
\hline
-l  & $L$   & 0.2 & \text{Signal spread low range} \\
-h  & $H$   & 0.6 & \text{Signal spread high range} \\
-v  & $V$   & 0.3 & \text{Pass value} \\
-m  & $M$   & 0.3 & \text{Signal quality floor} \\
-mc & $M_c$ & 0.2 & \text{Minimum signal quality} \\
-ml & $M_l$ & 150 & \text{Minimum signal size} \\
\hline
\end{tabular}
\end{center}
\caption{Parameters, default values and descriptions}
\label{table:defvalues}
\end{figure}

The filter has six parameters, summarized in Table~/ref{table:defvals}.

If the signals for a specific mRNA are all very similar, it is
probable that the weaker signals are weak only because of a few
mismatches that break 20-mers.  In this case, we cannot reliably pick
the signals that are true, and should consider all of them.

On the other hand, if there is a large range in the quality of signals,
we can safely discard low scoring signals, and still be confident that
we will find the good stuff.

Therefore, the filter will discard no signals if the range in quality
values is small, and will gradually discard more, proportional to the
range.  So that we don't discard too much, we limit the increase in
filtering to $V$ (0.3).
\begin{align*}
h &= bestCoverage - worstCoverage \\
p &= \begin{cases}
     0.0 & \text{if $h \le L$} \\
     V * \frac{h-L}{H-L} & \text{if $L < h < H$} \\
     V   & \text{if $H \le h$}
     \end{cases} \\
c &= min(worstCoverage + p \cdot h, M)
\end{align*}

\begin{figure}
\begin{center}
\epsfig{figure=mRNAfilt.eps, silent=, width=4.5in}
\end{center}
\caption{The $p$ curve.}
\label{fig:pcurve}
\end{figure}

$p$ is the amount of filtering, ranging from minimum (0.0) to maximum
($V$, a parameter).

The $c$ value computed above is the filtering threshold.  Signals with
coverage below $c$ are considered weak, and are discarded.

If the score range is small ($\le L$), then $c$ will be
$worstCoverage$, and we do no filtering.  If the score range is large
($\ge H$), then $c$ will be $M$ of the best score.  $c$ is the minimum
coverage that will be accepted.  It is derived from the range of
scores, not the number of scores.

Finally, it is possible that {\em all} signals are good.  If we used the
above filtering we would be discarding the low scoring (but still valid)
signals.  To overcome this, absolute limits $M_c$ and $M_l$ are enforced.

A signal is saved if both of the following conditions are met:
\begin{enumerate}
\item ($c <= coverage$)
\item ($M_c <= coverage$) or ($M_l <= coveredBases$)
\end{enumerate}

\section{Results}
This filter is overly permissive, throwing out only signals that are
obviously garbage.

\end{document}
