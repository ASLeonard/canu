\documentclass[twoside, twocolumn, 10pt]{article}
\usepackage{amsmath,amssymb}
\usepackage{moreverb}
\usepackage{fancyheadings}
\usepackage{ulem}
\usepackage{parskip}
\usepackage{calc,ifthen,epsfig}
\sloppy

%  A few float parameters
%
\renewcommand{\dbltopfraction}{0.9}
\renewcommand{\dblfloatpagefraction}{0.9}
%\renewcommand{\textfraction}{0.05}


\begin{document}

%  See page 63-64, LaTeX Companion
%
% leftmargin controls the left margin for EVERYTHING in the list!
%
\newcommand{\entrylabel}[1]{\mbox{\texttt{#1:}}\hfil}
\newenvironment{entry}
  {\begin{list}{}%
   {\renewcommand{\makelabel}{\entrylabel}%
    %\setlength{\leftmargin}{1.5in}%
   }}
{\end{list}}

% The first parbox width controls the indent on the first text line
% The makebox width seems to do nothing.
\newcommand{\Lentrylabel}[1]{%
  {\parbox[b]{0pt}{\makebox[0pt][l]{\texttt{#1:}}\\}}\hfil\relax}
\newenvironment{Lentry}
  {\renewcommand{\entrylabel}{\Lentrylabel}\begin{entry}}
  {\end{entry}}

\title{ESTmapper documentation}
\author{
Liliana Florea\thanks{liliana.florea@celera.com},
Brian P. Walenz\thanks{brian.walenz@celera.com}}

\maketitle

\pagestyle{fancy}

\rhead[]{}
\chead[ESTmapper]{ESTmapper}
\lhead[\today]{\today}

\normalem

\newcommand{\ESTmapper}{{\sc ESTmapper\ }}

\begin{abstract}
The gory details of the \ESTmapper process is described.






\subsection{Hash Function Definitions}

In the discussion that follows, let $A$ be an encoded mer, $H$ be the
hashed value of the mer, and $C$ be the check value.  $A$ is $m$ bits
wide, $H$ is $h$ bits wide and $C$ is $m-h$ bits wide.

Our hash and check functions must satisfy the following properties:
\begin{align*}
f_H &: m \rightarrow h \\
f_C &: m \rightarrow c \\
f_R &: h \times c \rightarrow m
\end{align*}
such that $f_R(f_H(A), f_C(A)) = A$.

Furthermore, $f_H$ should be a good hash function, {\bf whatever that means}.
The functions are explained in Section~\ref{sec:hashfcn}.

\subsection{existDB}

The {\tt existDB} will tell us if a mer exists in a sequence.

We can build the structure in Figure~\ref{fig:hashstruct} in five
steps, using $\theta(2 \cdot 2^h + 2 \cdot n)$ time and no temporary
space.

\begin{enumerate}
\item
Allocate and zero $2^h$ integers for the hash table.
\item
Count the size of each bucket: hash each mer, increment the size of
that bucket.  This can be done using the space for the hash table.
Also count the number of mers.
\item
Allocate $n$ bucket entries, one for each mer.  There is no need to
initialize these.
\item
Make the hash table entry $i$ point to the start of bucket $i$.  Note
that the hash table entry $i+1$ can be used to find the end of bucket
$i$.
\item
Rehash each mer, inserting the check value into the next available
bucket entry (use the hash table to keep track of the next available
entry).  The buckets contain all the mers after this step, and the
hash table is off by one -- entry $i$ points to the start of bucket
$i+1$.  If we offset the start of the table, we can fix this in $O(1)$
time.
\end{enumerate}

We assume that the input sequence does not contain duplicate mers.  If
it does, we should remove them from the table.

\subsection{positionDB}

We can extend the {\tt existDB} to store position information by
storing, in the bucket entry, either the position of the mer (if there
is exactly one copy) or a pointer to a list of positions.

Unlike the {\tt existDB} we now need to remove duplicate mers from the
table.

\begin{enumerate}
\item
count bucket size - this overcounts the true size; it counts
duplicates
\item
allocate counting buckets - build a list of hashed mers and the position
that they occur.
\item
sort each bucket
\item
allocate the final hash table, buckets and position lists
{\bf XXX} can we reuse the hash table and bucket space??
\item
copy the counting buckets into the final structure.  mers that occur
exactly once have their position stored in the bucket entry.  mers
that occur more than once have a pointer to the position list
placed in the bucket entry.
\end{enumerate}

\subsection{A Good Hash Function}
\label{sec:hashfcn}

A simple hash function would be to use the highest $h$ bits of the
encoded mer as the hash, and use the lowest $m-h$ bits of the mer
as the check.  Unfortunately, this is a very poor hash function ---
the hash function is strongly correlated with the input mer. {\bf needs more blah blah}

A better hash function would first ``scramble'' the bits in the mer to
break the correlation between the input and the output.

In the discussion that follows, let $A$ be an encoded mer, $H$ be the
hashed value of the mer, and $C$ be the collision resolution value.
$A$ is $m$ bits wide, $H$ is $h$ bits wide and $C$ is $m-h$ bits wide.

We want to find functions 
$f_H : m \rightarrow h$,
$f_C : m \rightarrow c$,
$f_R : h \times c \rightarrow m$
such that
\begin{align*}
f_H(A) =& H \\
f_C(A) =& C \\
f_R(H,C) =& A
\end{align*}
Furthermore, $f_H$ should be a good hash function.

We specify $f_H$ and $f_C$ by specifying each bit in the output.
\begin{align*}
H_i &= A_{i} \oplus A_{i-\frac{m-h}{2}} \oplus A_{i+m-h}, \text{ for } 1 \le i \le h \\
C_i &= A_{i}, \text{ for } 1 \le i \le m-h
\end{align*}
Likewise, $f_R$ can be expressed as
\begin{align*}
A_i &= 
\begin{cases}
C_i & 1 \le i \le m-h \\
A_{i-m+h} \oplus A_{i-\frac{m+h}{2}} \oplus H_{i-m+h} & m-h < i \le m
\end{cases}
\end{align*}
In C code
\begin{verbatim}
u64bit  fH(u64bit A) {
  return(((A) ^
          (A >> (m-h)/2) ^
          (A >> (m-h))) & MASK(h));
}

u64bit  fC(u64bit A) {
  return(A & MASK(m-h));
}
\end{verbatim}
where {\tt u64bit} is a 64-bit unsigned integer type.  The code for
$f_R$ is non-trivial, and is not needed.


\end{document}

